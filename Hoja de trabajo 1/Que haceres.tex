\documentclass[12pt, a4paper]{article} 
  \usepackage[left=15mm, top=0.5in, bottom=5in]{geometry}
  \usepackage[utf8]{inputenc}
  \usepackage{hyperref}
  \author{Juan Diego Vega Ruiz}
  \date{25 de Enero de 2018}
  
\begin{document} 
  \title{Hoja de Trabajo No.1} 
   \maketitle    
   Correo del alumno: vega171166@unis.edu.gt \\   
   Usuario de github: diegos99 \\
   Link del Repositorio: \href{https://github.com/diegos99/Informatica2-Tareas}{https://github.com/diegos99/Informatica2-Tareas}


    \section{"Que Hacer":}
        \begin{itemize}
            \item string \textbf{\emph{Hacer}} (nombre de la actividad).
            \item date \textbf{\emph{Día}} (día de la semana en que se realizara la actividad).
            \item time \textbf{\emph{Hora}} (momento especifico del día para realizar actividad).
            \item string \textbf{\emph{Lugar}} (espacio físico en donde se llevará a cabo la actividad).
            \item string \textbf{\emph{Duración}} (lapso de tiempo en que se realizará).
            \item string \textbf{\emph{Herramientas}} (Utensilios que ayudaran a llevar a cabo la actividad).
            \item string \textbf{\emph{Personas involucradas}} (personajes participando directamente en la actividad).
        \end{itemize}
    \section{"Que Haceres":}
        \begin{itemize}
            \item \textbf{Nuevo:} (agregar un nuevo que hacer a la lista).
            \item \textbf{Borrar:} (Borrar algún elemento de la lista).
            \item \textbf{Filtrar:} (búsqueda fácil de un “qué hacer” que se encuentra en la lista”).
            \item \textbf{Modificar:} (opción para cambiar el nombre de un que hacer, o bien, alguna propiedad).
            \item \textbf{Actualizar:} (opción que da un refresh a la lista de que hacer).
            \item \textbf{Contar:} (numera y muestra la cantidad de “que hacer” existen en la lista).       
        \end{itemize}
        
\end{document}